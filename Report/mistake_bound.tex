\section{Mistake Bound Model of Learning}

Consider an instance space consisting of {\bf integer points} on the
two dimensional plane $(x_1, x_2)$ with $-80 \leq x_1, x_2 \leq 80$.
Let $\mathcal C$ be a concept class defined on this instance space.
Each function $f_r$ in $\mathcal C$ is defined by a radius $r$ (with
$100 \leq r \leq 6400$) as follows:

\begin{equation}
f_r(x_1, x_2) = \left\{
    \begin{array}{rl}
      +1 & x_1^2 + x_2^2 \leq r^2;\\
      -1 & \mbox{otherwise}
    \end{array}
\right.
\label{eq:f_r}
\end{equation}


Our goal is to come up with a error-driven algorithm that will learn
the correct function $f \in \mathcal{C}$ that correctly classifies a
dataset.

\paragraph{Side notes}
\begin{enumerate}
\item Recall that a concept class is the set of functions from which
  the true target function is drawn and the hypothesis space is the
  set of functions that the learning algorithm searches over. In this
  question, both these are the same set.
\item Assume that there is no noise. That is, assume that the data is
  separable using the hypothesis class.
\end{enumerate}


\paragraph{Questions}

\begin{enumerate}


\item[1.] [5 points] Determine $|\mathcal{C}|$, the size of concept
  class.
  
\item[2.] [5 points] To design an error driven learning algorithm, we
  should be able to first write down what it means to make a mistake.
  Suppose our current guess for the function is $f_r$ defined as in
  Equation \ref{eq:f_r} above. Say we get an input point $(x_1^t,
  x_2^t)$ along with its label $y^t$. Write down an expression (an
  equality or an inequality) in terms of $x_1^t$, $x_2^t$, $y^t$ and
  $r$ that checks whether the current hypothesis $f_r$ has made a
  mistake.

\item[3.] [10 points] Next, we need to specify how we will update a
  hypothesis if there is an error. Since $f_r$ is completely defined
  in terms of $r$, we only need to update $r$. How will you update $r$
  if there is an error? Consider errors for both positive and negative
  examples.

\item[4.] [20 points] Use the answers from the previous two steps to
  write a mistake-driven learning algorithm to learn the function.
  Please write the algorithm concisely in the form of pseudocode. What
  is the maximum number of mistakes that this algorithm can make on
  any dataset?

\item[5.] ({\bf For 6350 students})[15 points total] We have seen the
  Halving algorithm in class. The Halving algorithm will maintain a
  set of hypotheses consistent with all the examples seen so far and
  predict using the most frequent label among this set. Upon making a
  mistake, the algorithm prune at least half of this set. In this
  question, you will design and analyze a Halving algorithm for this
  particular concept space.

  \begin{enumerate}
  \item[a.] [5 points] The set of hypotheses consistent with all
    examples seen so far can be defined by storing only two integers.
    How would you do this?
  \item[b.] [5 points] How would you check if there is an error for an
    example $(x_1^t, x_2^t)$ that has the label $y^t$?
  \item[c.] [5 points] Write the full Halving algorithm for this
    specific concept space. (Do not write the same Halving algorithm
    we saw in class. You need to tailor it to this problem.) What is
    its mistake bound?
  \end{enumerate}

\end{enumerate}


%%% Local Variables:
%%% mode: latex
%%% TeX-master: "hw"
%%% End:
